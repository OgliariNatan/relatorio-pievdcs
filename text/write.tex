
%
%         Destinado a confecção do relatorio
%
%
%
% Variaveis 
\def\numero_instituicoes{11}
\def\tempo_secao{9000}
%\def\com_detalhes#1#2{#1 (numero #2)} para uso \com_detalhes{parametro1}{parametro2}




\section{Introdução}

\noindent \begin{minipage}[c]{0.6\textwidth}
  \vspace {1cm}
  A Plataforma Integrada de Informações para Enfrentamento à Violência Doméstica e Crimes Sexuais (PIEVDCS) é uma solução digital desenvolvida com base no \textit{framework} Django, no modelo SaaS (Software como Serviço). Seu objetivo é integrar instituições como segurança pública, Ministério Público, Poder Judiciário, Defensoria Pública e serviços municipais de assistência, promovendo agilidade e efetividade no atendimento às vítimas e concientização dos agressores.

  
\end{minipage}
\begin{minipage}[c]{0.4\textwidth}
  \captionof{figure}{logo PIEVDCS.}
  \includegraphics[width=\textwidth]{figure/logo_PIEVDCS.png}
  \label{figlogo_PIEVDCS}
  {\fontsize{10pt}{\baselineskip}\selectfont
    Fonte: O autor (2025)}
\end{minipage}
\\
\par Além de promover agilidade e efetividade no atendimento às vítimas, a plataforma representa uma ferramenta estratégica para romper o ciclo intergeracional da violência, que frequentemente se perpetua de pais para filhos. Estudos como o da Universidade Federal do Ceará, em parceria com a ONU Mulheres, revelam que quatro em cada dez mulheres que cresceram em lares violentos vivenciam o mesmo padrão na vida adulta, evidenciando a urgência de intervenções sistêmicas e integradas \cite{carvalho2017transmissao}. Ao conectar instituições e facilitar o fluxo de informações, a plataforma não apenas melhora o presente - ela transforma o futuro, criando condições para que novas gerações cresçam em ambientes mais seguros, protegidos e livres da normalização da violência doméstica.



\par O desenvolvimento da plataforma foi viabilizado por meio de verba oriunda do Poder Judiciário da Comarca de Maravilha/SC. Um aspecto singular do projeto é a participação direta de um reeducando do Presídio Regional de Maravilha como responsável técnico pelo desenvolvimento da solução. O reeducando, com formação técnica, atua sob supervisão institucional, aplicando seus conhecimentos de forma produtiva e qualificada, demonstrando que a ressocialização é viável e impactante.

\par Este relatório documenta o progresso técnico da implementação, destacando as funcionalidades já desenvolvidas, os métodos utilizados e as etapas pendentes até a entrega final.

\section{Violência Doméstica no Brasil: Panorama e Dados}

\par A violência doméstica contra a mulher constitui um grave problema de saúde pública e violação dos direitos humanos no Brasil. Segundo dados do Fórum Brasileiro de Segurança Pública (FBSP), em 2023 foram registradas 1.467 mortes violentas intencionais de mulheres, sendo 722 feminicídios \cite{fbsp2024}. A cada hora, aproximadamente 46 mulheres sofrem algum tipo de agressão no Brasil, totalizando mais de 400 mil casos de violência doméstica registrados anualmente.

\par A Lei nº 11.340/2006, conhecida como Lei Maria da Penha, representa um marco legal fundamental no enfrentamento à violência doméstica e familiar contra a mulher. A legislação reconhece cinco formas de violência: física, psicológica, sexual, patrimonial e moral, estabelecendo mecanismos de proteção e assistência às vítimas \cite{brasil2006}. Desde sua promulgação, observou-se um aumento significativo nas denúncias, reflexo tanto da maior conscientização social quanto da ampliação dos canais de atendimento.

\par Dados do Instituto Maria da Penha revelam que 89\% das vítimas de violência doméstica no Brasil são mulheres, sendo que 43\% dos casos ocorrem dentro de casa. O parceiro íntimo (companheiro, marido, namorado ou ex) é o principal agressor em 78\% dos casos registrados \cite{imp2023}. Além disso, 68\% das mulheres em situação de violência convivem com o agressor, evidenciando a complexidade das relações de dependência emocional, econômica e familiar.

\par O Anuário Brasileiro de Segurança Pública de 2024 aponta que as medidas protetivas de urgência (MPU) continuam sendo uma das principais ferramentas de proteção às vítimas. Em 2023, foram concedidas mais de 362 mil medidas protetivas no país, representando um aumento de 17\% em relação ao ano anterior \cite{fbsp2024}. No entanto, desafios persistem na efetividade do monitoramento e na articulação entre as instituições responsáveis pela proteção das vítimas.

\par A pandemia de COVID-19 agravou o cenário da violência doméstica no Brasil. Durante o período de isolamento social em 2020 e 2021, houve um aumento de 28\% nas denúncias de violência contra mulheres, segundo o Ligue 180 (Central de Atendimento à Mulher). O confinamento forçado intensificou a convivência com agressores e dificultou o acesso das vítimas às redes de proteção e denúncia \cite{fbsp2021}.

\par No contexto de Santa Catarina, dados da Secretaria de Estado da Segurança Pública mostram que em 2023 foram registrados 38.456 casos de violência doméstica, sendo 21.234 casos de lesão corporal dolosa e 15.879 casos de ameaça \cite{ssp_sc2024}. O estado tem implementado políticas públicas integradas, incluindo casas-abrigo, centros de referência e patrulhas especializadas no atendimento à mulher.

\par A violência psicológica, frequentemente subnotificada, afeta profundamente a saúde mental das vítimas. Estudos indicam que 85\% das mulheres em situação de violência doméstica desenvolvem transtornos psicológicos, como depressão, ansiedade e síndrome do pânico \cite{zancan2013}. A violência também impacta diretamente as crianças expostas ao ambiente violento, perpetuando o ciclo intergeracional já mencionado.

\par Diante desse cenário alarmante, a implementação de sistemas integrados de informação, como a PIEVDCS, torna-se essencial para aprimorar a coordenação entre as instituições, acelerar o atendimento às vítimas, monitorar a efetividade das medidas protetivas e produzir dados estratégicos para o planejamento de políticas públicas de prevenção e enfrentamento à violência doméstica.

\section{Tecnologias utilizadas}
\par A escolha das tecnologias adotadas neste projeto foi pautada, sobretudo, por seu caráter \textit{Open Source} (código aberto), o que proporciona significativa economia ao eliminar custos com licenciamento. Dessa forma, os investimentos se concentram exclusivamente em aspectos de infraestrutura, como servidores e equipe técnica especializada.
\par Além disso, o desenvolvedor considera que as informações sensíveis contempladas por esta plataforma devem permanecer sob responsabilidade do Estado, não devendo ser compartilhadas com empresas privadas, como serviços de hospedagem em nuvem ou soluções de inteligência artificial de terceiros. Ressalta-se, contudo, que, na fase de validação, a plataforma está hospedada em serviços de terceiros.

\subsection{Framework: Django}
\par Segundo \citeonline{django:2025}, Django é um framework de desenvolvimento web escrito em Python, amplamente reconhecido por acelerar a criação de aplicações robustas e escaláveis. Seu diferencial está na abordagem \textit{"batteries included"}, oferecendo nativamente funcionalidades essenciais como autenticação, painel administrativo, mapeamento objeto-relacional (ORM) e roteamento de URLs (\textit{Uniform Resource Locator}).
\par Outro fator decisivo na escolha do Django é sua natureza nativa em Python, o que possibilita fácil integração com bibliotecas de \textit{machine learning}\footnote{Aprendizado de máquina}, como \texttt{scikit-learn}, \texttt{TensorFlow}, \texttt{PyTorch}, \texttt{Ollama}, dentre outras. Isso viabiliza, em futuras versões do sistema, a implementação de modelos preditivos capazes de analisar parâmetros operacionais e sugerir ações estratégicas de maneira automatizada e inteligente.

\subsection{Banco de dados: PostgreSQL}
\par \citeonline{postgresql:2025} é um sistema de gerenciamento de banco de dados objeto-relacional de código aberto, com mais de 35 anos de desenvolvimento ativo, considerado um dos mais avançados e robustos do mercado. Conta com integração nativa ao Django.
\par Em termos de gerenciamento de dados, o PostgreSQL oferece suporte a extensões como PostGIS, que adiciona funcionalidades geoespaciais essenciais para a análise e visualização de dados geográficos, como mapas de ocorrências de violência doméstica. Além disso, sua arquitetura robusta e escalável permite lidar com grandes volumes de dados, garantindo integridade, segurança e desempenho, aspectos cruciais para uma plataforma que visa integrar múltiplas instituições e gerenciar informações sensíveis.

\subsection{Nginx}
\par O \citeonline{nginx:2025} é um servidor web de alto desempenho e proxy reverso, amplamente utilizado para servir aplicações web em ambientes de produção. No contexto deste projeto, o Nginx será configurado como servidor HTTP para a versão de produção, substituindo o servidor de desenvolvimento do Django. Suas principais vantagens incluem alto desempenho, baixo consumo de recursos, capacidade de lidar com grande número de conexões simultâneas e funcionalidades avançadas de balanceamento de carga e cache.

\subsection{Tailwind CSS}
\par O \citeonline{tailwindCSS:2025} é um framework de estilo baseado em classes utilitárias, que permite o desenvolvimento rápido de interfaces personalizadas e responsivas. Sua abordagem facilita a manutenção e reutilização de estilos, promovendo consistência visual e produtividade no desenvolvimento front-end, mantendo os padrões de desenvolvimento de Experiência do Usuário/Interface do Usuário (UX/UI).

\subsection{Django Channels e Redis}
\par Para implementação de comunicação em tempo real na plataforma, foi adotado o \textbf{Django Channels} \cite{django_channels:2025}, uma extensão do Django que adiciona suporte a protocolos assíncronos como WebSockets\footnote{protocolo que estabelece canais de comunicações bidirecionais e persistente entre um cliente e um servidor, permitindo a troca de dados em tempo real.}. Esta tecnologia permite o envio e recebimento de notificações instantâneas entre usuários e instituições, fundamentais para a coordenação eficiente das ações de enfrentamento à violência doméstica.
\par Como \textit{backend} de mensagens, utiliza-se o \textbf{Redis} \cite{redis:2025}, um armazenamento de estrutura de dados em memória, que atua como \textit{broker} de mensagens garantindo alta performance e confiabilidade na entrega de notificações em tempo real. O sistema foi configurado com capacidade de 1500 conexões simultâneas e tempo de expiração de 10 segundos, assegurando escalabilidade e eficiência.

\subsection{Smart Selects}
\par A biblioteca \textbf{Smart Selects} \cite{smart_selects:2025} foi integrada ao projeto para implementar campos de seleção encadeados (\textit{chained selects}) nos formulários Django. Esta funcionalidade é essencial para garantir a consistência de dados, especialmente na seleção de municípios baseada no estado previamente escolhido, e na vinculação de municípios às suas respectivas comarcas do Poder Judiciário.

\subsection{Chart.js}
\par \citeonline{Chartjs:2025} é uma biblioteca JavaScript especializada na renderização de gráficos interativos e responsivos. Utilizada para a visualização dos dados gerados pela aplicação, permite criar diferentes tipos de gráficos (como barras, linhas, setores dentre outras opções), proporcionando clareza e insights para os usuários.
\par A biblioteca foi criada e mantida pelos \href{https://github.com/chartjs/Chart.js/graphs/contributors}{Voluntários}. \\
Link: (https://github.com/chartjs/Chart.js/graphs/contributors).

\subsection{Leaflet.js com OpenStreetMap}
\par \citeonline{Leafletjss:2025} é uma biblioteca JavaScript leve para a criação de mapas interativos. Em conjunto com a plataforma \href{https://www.openstreetmap.org/#map=7/-26.613/-50.746}{OpenStreetMap}, oferece uma base cartográfica gratuita e de alta qualidade. No contexto do projeto, foi empregada para a representação geoespacial de dados, com funcionalidades como marcação de locais, camadas e interações de usuário.

\subsection{jQuery}\label{jQuery}
\par \citeonline{jquery:2025} é uma biblioteca JavaScript amplamente utilizada para simplificar operações no DOM\footnote{Document Object Model (Modelo de Objeto de Documento) permite alterar parte do conteudo da página web sem a necessidade de recarregar toda a página.}, gerenciamento de eventos e requisições assíncronas (AJAX)\footnote{(Asynchronous JavaScript and XML) Permite comunicação em segundo plano com o servidor sem recarregar a página.}. No projeto, ela contribui para tornar a navegação mais fluida e as interações mais dinâmicas, além de facilitar a integração com outras bibliotecas e funcionalidades da aplicação.

\subsection{Ollama: Execução Local de Modelos de Linguagem}
\par A execução de Modelos de Linguagem de Grande Escala (LLMs) será realizada utilizando a plataforma \texttt{Ollama}, que permite rodar modelos como LLaMA, Mistral e Gemma diretamente no ambiente local, sem necessidade de conexão com servidores externos. Segundo \citeonline{ollama:2025}, a proposta da ferramenta é “facilitar a execução de grandes modelos de linguagem localmente”, o que contribui para maior privacidade e controle sobre os dados processados.
\par A execução dos modelos será realizada com suporte de uma GPU \texttt{NVIDIA RTX 2000 Ada Generation}, equipada com 16GB de memória dedicada, DDR6, 128bits. Embora não seja a mais potente da linha, essa placa gráfica oferece recursos suficientes para testes locais com modelos otimizados, como os disponíveis na plataforma Ollama. Seu uso viabiliza a aceleração das inferências, mesmo que com tempos de resposta mais elevados em modelos de maior complexidade.

% filepath: /media/ogliari/backup_ssd/GitHub/relatorio-pievdcs/text/write.tex
\section{Níveis de Prontidão Tecnológica (TRL)}\label{trl}

\par A escala de Níveis de Prontidão Tecnológica (\textit{Technology Readiness Level} - TRL) é uma metodologia sistemática utilizada para avaliar o grau de maturidade de uma tecnologia específica, desde sua concepção teórica até sua implementação operacional completa. Desenvolvida originalmente pela NASA na década de 1970, a escala TRL tornou-se um padrão internacional para avaliação de projetos tecnológicos \cite{trl:2025}.

\par A metodologia TRL é estruturada em nove níveis progressivos, permitindo uma avaliação objetiva do estágio de desenvolvimento e dos riscos associados à implementação de novas tecnologias. Cada nível representa um marco específico no processo de maturação tecnológica, desde a pesquisa básica até a operação comprovada em ambiente real.

\subsection{Classificação dos Níveis TRL}

\begin{description}
    \item[TRL 1 -- Princípios básicos observados e reportados] Fase inicial de pesquisa científica, onde os princípios fundamentais são identificados e documentados teoricamente.
    
    \item[TRL 2 -- Formulação de conceitos tecnológicos e/ou de aplicação] Desenvolvimento conceitual da tecnologia, com proposição de possíveis aplicações práticas.
    
    \item[TRL 3 -- Estabelecimento de função crítica de forma analítica ou experimental e/ou prova de conceito] Realização de estudos analíticos e experimentações iniciais para validar conceitos críticos da tecnologia.
    
    \item[TRL 4 -- Validação funcional dos componentes em ambiente de laboratório] Testes controlados em laboratório para verificar o funcionamento dos componentes individuais do sistema.
    
    \item[TRL 5 -- Validação das funções críticas dos componentes em ambiente relevante] Verificação do desempenho dos componentes em condições que simulam o ambiente operacional real.
    
    \item[TRL 6 -- Demonstração de funções críticas do protótipo em ambiente relevante] Teste de um protótipo do sistema completo em ambiente que representa adequadamente as condições operacionais.
    
    \item[TRL 7 -- Demonstração de protótipo do sistema em ambiente operacional] Validação do protótipo em ambiente operacional real, com usuários finais e condições reais de uso.
    
    \item[TRL 8 -- Sistema qualificado e finalizado] Sistema completo testado e aprovado em todas as suas funcionalidades, pronto para implementação em larga escala.
    
    \item[TRL 9 -- Sistema operando e comprovado em todos os aspectos de sua missão operacional] Tecnologia totalmente madura, operando com sucesso em ambiente real e atendendo plenamente aos requisitos operacionais.
\end{description}


\section{Níveis de Prontidão Tecnológica (TRL)}\label{trl}

\par A escala de Níveis de Prontidão Tecnológica (\textit{Technology Readiness Level} - TRL) é uma metodologia sistemática utilizada para avaliar o grau de maturidade de uma tecnologia específica, desde sua concepção teórica até sua implementação operacional completa. Desenvolvida originalmente pela NASA na década de 1970, a escala TRL tornou-se um padrão internacional para avaliação de projetos tecnológicos, sendo posteriormente consolidada pela norma ISO 16290:2013 e adaptada no Brasil pela NBR ISO 16290:2015 \cite{trl:2025}.

\par A metodologia TRL é estruturada em nove níveis progressivos, permitindo uma avaliação objetiva do estágio de desenvolvimento e dos riscos associados à implementação de novas tecnologias. Cada nível representa um marco específico no processo de maturação tecnológica, desde a pesquisa básica até a operação comprovada em ambiente real. Para facilitar o entendimento, estes níveis são agrupados em cinco categorias principais relacionadas ao ciclo de vida de projetos de inovação tecnológica: pesquisa básica, pesquisa aplicada, desenvolvimento experimental, industrialização e produção/comercialização.

\subsection{Classificação dos Níveis TRL}

\begin{description}
    \item[TRL 1 -- Princípios básicos observados e reportados] Fase inicial de pesquisa científica, onde os princípios fundamentais são identificados e documentados teoricamente. \textit{(Grupo: Pesquisa Básica)}
    
    \item[TRL 2 -- Formulação de conceitos tecnológicos e/ou de aplicação] Desenvolvimento conceitual da tecnologia, com proposição de possíveis aplicações práticas. \textit{(Grupo: Pesquisa Básica)}
    
    \item[TRL 3 -- Estabelecimento de função crítica de forma analítica ou experimental e/ou prova de conceito] Realização de estudos analíticos e experimentações iniciais para validar conceitos críticos da tecnologia. \textit{(Grupo: Pesquisa Aplicada)}
    
    \item[TRL 4 -- Validação funcional dos componentes em ambiente de laboratório] Testes controlados em laboratório para verificar o funcionamento dos componentes individuais do sistema. \textit{(Grupo: Pesquisa Aplicada)}
    
    \item[TRL 5 -- Validação das funções críticas dos componentes em ambiente relevante] Verificação do desempenho dos componentes em condições que simulam o ambiente operacional real. \textit{(Grupo: Desenvolvimento Experimental)}
    
    \item[TRL 6 -- Demonstração de funções críticas do protótipo em ambiente relevante] Teste de um protótipo do sistema completo em ambiente que representa adequadamente as condições operacionais. \textit{(Grupo: Desenvolvimento Experimental)}
    
    \item[TRL 7 -- Demonstração de protótipo do sistema em ambiente operacional] Validação do protótipo em ambiente operacional real, com usuários finais e condições reais de uso. \textit{(Grupo: Industrialização)}
    
    \item[TRL 8 -- Sistema qualificado e finalizado] Sistema completo testado e aprovado em todas as suas funcionalidades, pronto para implementação em larga escala. \textit{(Grupo: Industrialização)}
    
    \item[TRL 9 -- Sistema operando e comprovado em todos os aspectos de sua missão operacional] Tecnologia totalmente madura, operando com sucesso em ambiente real e atendendo plenamente aos requisitos operacionais. \textit{(Grupo: Produção e Comercialização)}
\end{description}

\subsection{Classificação da escala de nível de prontidão tecnológica (TRL)} \ref{trl}

\par O projeto PIEVDCS encontra-se atualmente transitando entre os níveis \textbf{TRL 5 e TRL 6}, com perspectiva de alcançar o \textbf{TRL 7} ao final da fase de validação atual. Esta classificação se justifica pelos seguintes marcos já atingidos e em desenvolvimento:

\begin{itemize}
    \item \textbf{TRL 1 e 2 (Concluídos - Pesquisa Básica):} Os princípios fundamentais foram estabelecidos através do levantamento de requisitos com as instituições participantes, identificação das necessidades de integração entre os sistemas de justiça, segurança pública e serviços municipais, e formulação do conceito de plataforma integrada para enfrentamento à violência doméstica.
    
    \item \textbf{TRL 3 e 4 (Concluídos - Pesquisa Aplicada):} Os conceitos tecnológicos foram estabelecidos e validados em ambiente de desenvolvimento. A arquitetura Django com PostgreSQL, sistema de autenticação customizado baseado em instituições e integração entre módulos foram testados com sucesso em laboratório (ambiente de desenvolvimento local). Validação funcional dos componentes críticos: sistema de cadastro de vítimas/agressores, formulários de medidas protetivas, sistema de notificações em tempo real via WebSockets e dashboards estatísticos.
    
    \item \textbf{TRL 5 (Concluído - Desenvolvimento Experimental):} As funções críticas do sistema foram validadas em ambiente relevante através do servidor de desenvolvimento (\texttt{10.40.22.46:8000}) acessível pela rede interna do estado catarinense. Testes realizados com dados reais demonstraram a viabilidade técnica da integração entre as \numero_instituicoes{} instituições participantes, incluindo fluxos completos desde o registro de ocorrências até a emissão de medidas protetivas.
    
    \item \textbf{TRL 6 (Em andamento - Desenvolvimento Experimental):} O protótipo funcional está sendo demonstrado em ambiente operacional relevante através da plataforma Hostinger (\href{https://www.redecontraaviolencia.org}{https://www.redecontraaviolencia.org}), permitindo acesso externo controlado e testes com dados fictícios por potenciais usuários finais das diferentes instituições. Esta fase inclui validação de usabilidade, performance sob carga e refinamento de interfaces baseado em \textit{feedback} dos usuários.
    
    \item \textbf{TRL 7 (Meta do projeto piloto - Industrialização):} A fase final do projeto piloto consistirá na validação do protótipo com usuários reais das instituições participantes (Poder Judiciário, Ministério Público, Defensoria Pública, Polícia Militar, Polícia Civil, Polícia Penal e serviços municipais) em ambiente operacional real, porém ainda com dados controlados e supervisionados. Esta é a meta máxima estabelecida para o escopo atual do projeto, marcando a transição da fase de desenvolvimento experimental para a fase de industrialização.
\end{itemize}

\par É importante destacar que o projeto \textbf{não avançará para os níveis TRL 8 e 9} (fases de industrialização final e produção/comercialização) nesta etapa piloto. A transição para esses níveis superiores -- que representam sistema qualificado, finalizado e operando em plena capacidade produtiva -- demandaria:

\begin{enumerate}
    \item Aprovação formal de todas as instituições participantes após testes extensivos com dados reais;
    \item Integração completa com sistemas legados já em operação (e-eproc do Poder Judiciário, sistemas do Ministério Público, SISP das forças de segurança);
    \item Certificações de segurança da informação e conformidade plena com a LGPD\footnote{Lei Geral de Proteção de Dados Pessoais (Lei nº 13.709/2018)}, incluindo auditorias independentes;
    \item Infraestrutura de produção robusta com garantias de alta disponibilidade (99,9\%), redundância geográfica e planos de contingência;
    \item Programa completo de treinamento e capacitação para todos os operadores das instituições participantes;
    \item Período estendido de operação assistida com suporte técnico dedicado e monitoramento contínuo de desempenho;
    \item Validação de escalabilidade para expansão regional (demais comarcas de Santa Catarina) ou nacional;
    \item Estabelecimento de governança institucional permanente e modelo de sustentabilidade financeira para manutenção evolutiva do sistema.
\end{enumerate}

\par A decisão estratégica de limitar o escopo ao \textbf{TRL 7} está alinhada com a natureza de \textbf{projeto piloto} da iniciativa, conforme estabelecido pela NBR ISO 16290:2015. Esta abordagem permite validação técnica e operacional em escala controlada antes de uma possível expansão futura, minimizando riscos técnicos, organizacionais e financeiros. O modelo piloto possibilita ajustes iterativos baseados em \textit{feedback} real dos usuários, demonstra a viabilidade da solução em condições operacionais reais e gera evidências concretas de valor agregado antes de investimentos mais substanciais em infraestrutura e implantação em larga escala.

\par O acompanhamento sistemático da evolução do projeto através da metodologia TRL, conforme padronizada pela NBR ISO 16290:2015, permite aos \textit{stakeholders} (gestores públicos, instituições participantes e órgãos de fomento) compreender claramente o estágio atual de maturidade da tecnologia, os riscos associados a cada fase de transição e os próximos passos necessários para avançar rumo a uma solução completamente operacional e escalável no combate à violência doméstica e crimes sexuais.

\section{Implementação para testes e validações}
\par Para os testes iniciais e validações, as informações serão geradas automaticamente através de \textit{script python} \ref{script}, por se tratar de dados vulneráveis, não será utilizados informações reais nesta etapa do processo.

\subsection{Dos Scripts}\label{script}
\par Possui uma seção denominada de \textbf{automacoes}, esta pasta é responsável pela centralização de todos os \textit{script} de automações para a inserção de informações no banco de dados, sendo desde a configurações de Municípios, Comarcas até a inserção de valores fictícios para validação da plataforma com os usuários finais.
\par Funcionalidades já implementadas nos \textit{script}

\begin{description}
   \item [Cria Estado] Função que cria no banco de dados todos os estados da federação e um estado \textbf{"EX" - Extrangeiro} - Script para configuração da plataforma, pode ser executado n vezes, pois só cria se não existir.
   \item [Atribui Município] Script que atibui todos os municípios de cada respectivo estado, evitando duplicidade e relacionando com o estado pertinente, deverá ser executado após o "Cria Estado".
   \item [Cria Comarca] Funcionalidade que cria todas as comarcas do estado Catarinense com base no site do Tribunal de Justiça Catarinense, \href{https://www.tjsc.jus.br/paginas-das-comarcas}{https://www.tjsc.jus.br/paginas-das-comarcas} - Funcionalidade de configuração - a futuro, complementar para a inserção de todas as comarcas da federação com seus respectivos \textit{foro}. Função a ser executada após o "Atribui Município".
   \item [Gera Vítimas] Cadastra aleatoriamente vítimas geradas randomicamente. Pode ser utilizada quantas vezes for necessárias.
   \item [Gera Agressores] Cadastra aleatoriamente Agressores gerados randomicamente.
   \item [Cadastro de MP e MPU] Gera aleatoriamente Medidas protetivas e Medidas protetivas de urgência e cadastra na seção competente a \textbf{Defensoria Pública}.
   \item [Cria Grupo de Usuários] Cria grupos institucionais para que possa ser inserido usuários pertinentes as suas instituições e suas permissões. [\textit{CAPS, CRAS, CREAS, Secretaria de Saúde, Polícia Penal, Polícia Militar, Polícia Civil, Polícia Científica, Poder Judiciário, Ministério Público e Defensoria Pública}].
 \end{description}

%\lstinputlisting[language=python, caption={código externo}, label={cod:externo}]{main.c} %Busca os codigos na pasta /cod


\subsection{Hostinger}
\par A Hostinger é uma empresa global de hospedagem de sites que oferece serviços como hospedagem compartilhada, VPS (Servidor Virtual Privado), hospedagem em nuvem, registro de domínios e criador de sites. Fundada em 2004, a empresa se destacou no mercado por oferecer soluções acessíveis, com boa performance e suporte técnico multilíngue. Atualmente, a Hostinger atende milhões de clientes em todo o mundo e se tornou uma das principais opções para desenvolvedores, empreendedores e pequenas empresas que buscam presença online confiável e de fácil gerenciamento~\cite{hostinger2025}.
\par Esta seção será subdividida em três partes: a aplicação da plataforma como serviço, o banco de dados em \textit{PostgreSQL} e o domínio de acesso via rede mundial de computadores.


\subsubsection{Aplicação SaaS}
\par A aplicação foi implementada em um servidor VPS (\textit{Virtual Private Server})\footnote{Servidor Virtual Privado: uma instância virtual isolada dentro de um servidor físico, com recursos próprios de CPU, RAM e armazenamento. Ideal para aplicações que exigem controle total do ambiente.} contratado junto à Hostinger, utilizando a infraestrutura como serviço (IaaS) oferecida pela plataforma. A instância VPS utilizada possui as seguintes configurações: \textbf{2 núcleos de CPU}, \textbf{8\,GB de memória RAM} e \textbf{100\,GB de armazenamento em disco SSD}.
\par O plano contratado teve início em \textbf{05 de agosto de 2025}, com vigência de \textbf{1 ano}, totalizando um custo de \textbf{R\$563,88} para o período. Essa infraestrutura garante a escalabilidade e estabilidade necessárias para o funcionamento da aplicação, permitindo seu acesso contínuo via rede pública.
\par A seguir, são descritos os principais recursos da VPS utilizada e o impacto de cada um no desempenho da aplicação:

\begin{itemize}
    \item \textbf{CPU (2 núcleos):} A quantidade de núcleos da CPU influencia diretamente na capacidade de processamento paralelo do servidor. Com dois núcleos, é possível atender múltiplas requisições simultâneas de forma eficiente, garantindo responsividade adequada da aplicação em cenários de uso médio/moderado.

    \item \textbf{Memória RAM (8\,GB):} A memória RAM é essencial para o carregamento de aplicações, bibliotecas, cache e o gerenciamento de múltiplas sessões de usuários. Com 8\,GB, a aplicação possui espaço suficiente para operar com fluidez, mesmo sob cargas de trabalho maiores, como múltiplos acessos simultâneos.

    \item \textbf{Armazenamento (100\,GB SSD):} O espaço em disco é utilizado para armazenar o código da aplicação, arquivos estáticos, banco de dados local e arquivos temporários. O uso de disco SSD, em comparação com HDs tradicionais, oferece maior velocidade de leitura e escrita, resultando em tempos de resposta mais rápidos e menor latência nas operações de I/O.
\end{itemize}


\subsubsection{O Banco de dados}
\par Com relação ao banco de dados, a sua configuração ficou a cargo do PostgreSQL 14, com acesso local e configurações passadas para o Django através de variáveis de ambientes.

\subsubsection{O domínio}
\par O domínio a ser utilizado é o \href{https://www.redecontraaviolencia.org}{https://www.redecontraaviolencia.org}, seu registro foi feito no dia 05 de agosto de 2025, terá validade de 1 ano (05/08/2026), o custo do domíno foi de R\$51,08 e o registro foi pela empresa \href{https://www.hostinger.com/br}{Hostinger}. Podendo a aplicação ser acessada através de seu IP\footnote{IP é como se fosse o endereço da sua casa, mas na internet.} (Protocolo de Internet), e de sua porta de serviço. \href{62.72.9.77:8000}{62.72.9.77:8000}.
\par Para a aplicação foi configurado um certificado TLS (Transport Layer Security), garantindo a segurança na comunicação entre o servidor e os usuários. O certificado é válido até 12/11/2025.

\subsection{Do Desenvolvimento}
\par Durante o desenvolvimento da aplicação — entendido como o ambiente onde a variável \texttt{DEBUG = True} está configurada no arquivo \texttt{.env} —, é comum utilizar recursos de depuração de código. Para tal, o desenvolvedor implementou um mecanismo de impressão no terminal de informações consideradas relevantes. Esse artifício permite que, quando a solução for implementada em produção, a depuração seja desativada de forma controlada, utilizando o seguinte padrão:

\begin{algorithm}[H]
    \caption{Uso de Decoradores para Depuração}
    \label{alg:depuracao_uso}
    
   \begin{lstlisting}[language=Python, basicstyle=\small\ttfamily]
      # Outras importacoes
      """ Configuraçao de decoradores para debug """
      import os

      var_debug = os.getenv('DEBUG', False) #Carrega apenas a variavel de debug

      if var_debug == 'True':
         from MAIN.decoradores.calcula_tempo import calcula_tempo, calcula_tempo_fun
         checked_debug_decorador = calcula_tempo
         checked_debug_decorador_fun = calcula_tempo_fun
         
      else:
         checked_debug_decorador = None
         checked_debug_decorador_fun = None

      """ Fim da configuraçao de decoradores para debug """
      
            
      # No inicio de cada visualização
      @checked_debug_decorador
      @login_required(login_url=reverse_lazy('login'))
      @grupos_permitidos(['instituicao_autorizadas', 'outras_instituicoes_autorizadas'])
      def funcao(request):
         pass

      # No inicio de cada função
      @checked_debug_decorador_fun
      @login_required(login_url=reverse_lazy('login'))
      @grupos_permitidos(['instituicao_autorizadas', 'outras_instituicoes_autorizadas'])
      def funcao_qualquer(parametros):
         # code
         # Para exibir impressões no terminal
         if var_debug == 'True':
            print("Impressão de depuração")
         # code
   \end{lstlisting}

   {\fontsize{10pt}{\baselineskip}\selectfont
   Fonte: O autor (2025)}
\end{algorithm}

\par Ao utilizar esse artifício de depuração (\ref{alg:depuracao_uso}), o desenvolvedor pode aplicar os decoradores \texttt{@checked\_debug\_decorador} ou \texttt{@checked\_debug\_decorador\_fun} para medir o tempo de execução de cada visualização (\textit{view}) ou função específica. Além disso, dentro do código, é possível inserir comandos de impressão condicionais que serão executados apenas quando a variável de depuração estiver ativa. Essa abordagem permite um monitoramento detalhado do desempenho e comportamento do código durante o desenvolvimento, facilitando a identificação e resolução de problemas sem impactar o ambiente de produção.


\section{Métodos}

A estrutura de desenvolvimento adotada é modular, baseada em aplicações Django (apps), com automações para carga de dados, visualização via dashboards\footnote{Os dashboards são interfaces gráficas que permitem a visualização de dados de forma interativa e dinâmica.} e controle granular de permissões por perfil institucional.

\subsection{Estrutura de Aplicações Django}
\par O projeto está organizado em aplicações Django independentes e interconectadas, seguindo o princípio de separação de responsabilidades:

\begin{description}
   \item[\texttt{MAIN}] Aplicação principal que gerencia a estrutura base do projeto, incluindo configurações globais, roteamento de URLs, página inicial com conteúdos dinâmicos e cálculo de variáveis estatísticas compartilhadas entre módulos.
   
   \item[\texttt{usuarios}] Gerencia autenticação, autorização e perfis de usuários customizados (\texttt{CustomUser}). Implementa grupos personalizados com permissões específicas por instituição.
   
   \item[\texttt{sistema\_justica}] Centraliza os módulos do sistema de justiça, subdividido em:
   \begin{itemize}
      \item \textbf{Poder Judiciário}: Gestão de comarcas, cadastro de vítimas e agressores, integração com chatbot de IA;
      \item \textbf{Ministério Público}: Interface específica para promotores de justiça;
      \item \textbf{Defensoria Pública}: Formulários de solicitação de medidas protetivas de urgência (MPU) com mais de 30 campos detalhados sobre tipos de violência.
   \end{itemize}
   
   \item[\texttt{seguranca\_publica}] Engloba as instituições de segurança pública:
   \begin{itemize}
      \item \textbf{Polícia Penal}: Registro de atendimentos a agressores (assistência social, psicologia, grupos reflexivos);
      \item \textbf{Polícia Civil}: Registro de ocorrências, investigações e flagrantes;
      \item \textbf{Polícia Militar}: Registro de atendimentos por tipo de patrulha (Radio Patrulha, Emergência 190, Policiamento Montado, etc.).
   \end{itemize}
   
   \item[\texttt{municipio}]\footnote{No inicio possuíamos o conselho tutelar, no entanto, como não se trata de uma equipe técnica, mas sim de conselheiros eleitos pela sociedade, optamos por não incluí-los como um grupo institucional, devido a informações criticas contidas na plataforma.} Abrange serviços municipais de assistência:
   \begin{itemize}
      \item CRAS (Centro de Referência de Assistência Social);
      \item CREAS (Centro de Referência Especializado de Assistência Social);
      \item CAPS (Centro de Atenção Psicossocial);
      \item Secretaria da Saúde.
   \end{itemize}
   
   \item[\texttt{mensageria}] Sistema de notificações em tempo real utilizando Django Channels e WebSockets, permitindo envio de mensagens para usuários individuais ou grupos institucionais, com classificação de prioridade (normal, urgente, crítica) e status de leitura.
\end{description}

\subsection{Técnicas e Práticas de Desenvolvimento}

Algumas técnicas adotadas incluem:
\begin{itemize}
\item \textbf{Automação de Dados de Teste}: Scripts Python localizados no diretório \texttt{automacoes/} para geração automática de dados fictícios (vítimas, agressores, municípios, comarcas, medidas protetivas), permitindo validação realista da plataforma sem comprometer dados reais. Exemplos incluem \texttt{gera\_formularios\_mp.py}, \texttt{atribui\_municipio.py} e \texttt{cria\_grupos\_usuarios.py}.

\item \textbf{Modelos de Dados Complexos}: Implementação de relacionamentos avançados entre entidades utilizando:
   \begin{itemize}
      \item \texttt{ChainedForeignKey} para seleções dependentes (município baseado em estado);
      \item \texttt{ManyToManyField} para relações múltiplas (agressores em grupos de atendimento);
      \item Cálculo automático de idade baseado em data de nascimento;
      \item Formatação automática de CPF e validações customizadas.
   \end{itemize}

\item \textbf{Sistema de Notificações em Tempo Real}: Arquitetura baseada em WebSockets com Django Channels, permitindo:
   \begin{itemize}
      \item Envio de notificações para usuários individuais ou grupos institucionais;
      \item Classificação por prioridade (normal, urgente, crítica);
      \item Rastreamento de status (não lida, lida, arquivada);
      \item APIs REST para contadores e listagens de notificações recentes.
   \end{itemize}

\item \textbf{Visualizações Dinâmicas}: Criação de dashboards interativos com \textit{Chart.js} para análise de tipos de violência, medidas protetivas por comarca, e mapas geográficos com \textit{Leaflet.js} para visualização espacial de ocorrências.

\item \textbf{Controle Granular de Permissões}: Sistema de grupos customizados com permissões específicas por instituição, implementando diferentes níveis de acesso (visualização, adição, modificação) conforme o perfil institucional.

\item \textbf{Templates Responsivos}: Interface desenvolvida com \textit{TailwindCSS}, garantindo adaptabilidade a diferentes dispositivos (desktop, tablets, smartphones).

\item \textbf{Segurança Avançada}: 
   \begin{itemize}
      \item Proteção CSRF\footnote{Cross-Site Request Forgery (falsificação de solicitação entre sites)};
      \item Configuração de CORS para APIs;
      \item Sessões com tempo limite configurável (\tempo_secao{} segundos);
      \item Cookies HTTP-only e SameSite;
      \item Separação de variáveis sensíveis com \textit{Dotenv};
      \item Sistema de \textit{logging} customizado para auditoria.
   \end{itemize}

\item \textbf{Integração com IA Local}: Implementação e testes com modelos de linguagem via Ollama (\texttt{llama3:70b}, \texttt{mixtral}, \texttt{gpt-oss:120b}), com configurações em \texttt{settings.py} permitindo alternância entre modelos. Sistema de fallback com respostas demo quando Ollama não está disponível.

\item \textbf{Decoradores Customizados}: Implementação de decoradores Python para:
   \begin{itemize}
      \item Cálculo de tempo de execução de views (\texttt{@calcula\_tempo});
      \item Controle de acesso por grupo institucional (\texttt{@grupos\_permitidos}).
   \end{itemize}
\end{itemize}

\par O desenvolvimento da plataforma foi iniciado em ambiente Windows 11, com ambiente virtual Python. Atualmente esta sendo desenvolvido no sistema operacional com base Linux (Ubuntu 24.04.3 LTS), tal migração se justifica e se sustenta em virtude da sua flexibilidade, autonomia na configuração de ambientes complexos e implementações de inteligências artificiais em ambiente de desenvolvimento local.

\par Estão sendo implementados modelos de inteligência artificial utilizando o Ollama, com restrições específicas ao tema de violência doméstica e à Lei nº 11.340/2006 (Lei Maria da Penha). O objetivo é desenvolver uma assistente virtual capaz de oferecer atendimento qualificado às vítimas e ao público em geral, garantindo conformidade legal e segurança nas respostas.

\par Para fins de teste, está sendo utilizado o servidor de desenvolvimento do Django o (\textit{Django server}\footnote{Pode ser acessado dentro da rede do estado Catarinense atraves do 10.40.22.46:8000}). Na versão de produção, o servidor elencado foi o \textbf{Nginx}. O \textit{deploy} da aplicação esta sendo implementado na plataforma \textbf{Hostinger}, permitindo a validação da interface do usuário e de suas funcionalidades em ambiente real.


\section{Resultados}

A seguir estão listadas as principais funcionalidades já implementadas:

\begin{itemize}
    \item \textbf{Estrutura de dados completa}: Modelos Django para vítimas, agressores, filhos, usuários, estados, municípios, comarcas e formulários especializados;
    
    \item \textbf{Sistema de autenticação robusto}: Login customizado com usuários estendidos (\texttt{CustomUser}), CPF, telefone, data de nascimento, gênero, Foto de perfil e vinculação institucional;
    
    \item \textbf{Gestão de grupos e permissões}: Sistema automatizado de criação de grupos institucionais (\numero_instituicoes{} grupos) com permissões granulares específicas por app (Sistema de Justiça, Segurança Pública, Município) e por instituições;
    
    \item \textbf{Formulário de Medida Protetiva extenso}: Implementado na Defensoria Pública com mais de 35 campos detalhados sobre tipos de violência, condutas de controle, ameaças, uso de tecnologia/IA para manipulação e violência vicária;
    
    \item \textbf{Registro de Ocorrências e acompanhamentos}: Polícia Militar (com 14 tipos de patrulha), Polícia Civil (investigação e flagrantes), Polícia Penal (atendimentos individuais e em grupo com agressores);
    
    \item \textbf{Sistema de Notificações em Tempo Real}: WebSocket com Django Channels, envio para usuários/grupos, prioridades (Normal, Urgente, Crítica), status (Não lida, Lida, Arquivada);
    
    \item \textbf{Dashboards estatísticos públicos}: Gráficos de tipos de violência, medidas protetivas por comarca, mapas geográficos (Leaflet.js + OpenStreetMap) com dados anonimizados;
    
    \item \textbf{Interface institucional diferenciada}: Cada instituição possui home page customizada com contador de encaminhamentos, notificações e acesso aos formulários específicos;
    
    \item \textbf{Chatbot com IA}: Interface para interação com modelos Ollama, sistema de fallback com respostas demo, configuração flexível de host e modelo;
    
    \item \textbf{Scripts de automação}: Criação de estados, municípios, comarcas de SC, geração de 500+ registros fictícios de MPU, criação automática de grupos com permissões;
    
    \item \textbf{Documentação}: Instruções técnicas e README detalhado de cada aplicação Django;
    
    \item \textbf{Páginas Concluídas}: Login (com recuperação de senha), Página Inicial pública (conteúdos dinâmicos), Relatórios públicos, Home institucional das \numero_instituicoes{} instituições, Formulários de cadastro, Interface de notificações.
\end{itemize}

\par Incluo algumas telas da plataforma, a sequência das imagens são aleatórias.

\begin{figure}[H]
    \center
    \subfigure[ Tela de Login.\label{fig:login}]{\includegraphics[scale=0.5]{figure/login.png}}
    \subfigure[Página aberta, estilo one-page.\label{fig:one-page}]{\includegraphics[scale=.2]{figure/one-page_1.png}}
    \subfigure[Página aberta, estilo one-page 1.\label{fig:one-page1}]{\includegraphics[scale=.2]{figure/one-page_3.png}}
    \subfigure[Página aberta, estilo one-page 2.\label{fig:one-page2}]{\includegraphics[scale=.2]{figure/one-page_2.png}}
    \caption{Telas da plataforma, O autor}\label{fig:telas_plataforma}
\end{figure}
\par Foi desenvolvida uma seção pública de relatórios, permitindo que portais de notícias ou entidades interessadas acessem estatísticas em tempo real. As informações apresentadas são reais e baseadas nos registros da plataforma, mas tratadas com rigor técnico para garantir o anonimato das vítimas e envolvidos. As localizações exibidas são referenciadas a centros comunitários ou órgãos públicos de cada bairro, evitando a identificação direta dos locais de ocorrência.

\begin{figure}[H]
    \center
    \subfigure[Gráfico da Etinia e Classe Econômica.\label{fig:car_map}]{\includegraphics[scale=.2]{figure/info_3.png}}
    \subfigure[Gráfico do Grau de Instrução e Grau de Parentesco.\label{fig:car_map1}]{\includegraphics[scale=.2]{figure/info_1.png}}
    \subfigure[Frases Motivacionais e alguns indicadores.\label{fig:car_fras}]{\includegraphics[scale=.4]{figure/info_2.png}}
    \subfigure[Gráfico Geográfico dos tipos de vilência.\label{fig:car_map2}]{\includegraphics[scale=.5]{figure/cards_map.png}}
    \caption{Telas estatísticas, O autor}\label{fig:telas_plataforma1}
\end{figure}

\par Esta sendo criada telas para cada instituição que terá acesso a plataforma e cada instituição terá determinado acesso e cada funcionalidade. Os \textit{layout} das instituições ficará a cargo de cada representante da referida em ajustar com o desenvolvedor, para usabilidade e funcionalidades de cada instituição.

\begin{figure}[H]
    \center
    \subfigure[Seleção das instituições.\label{fig:selecao_insti}]{\includegraphics[scale=.3]{figure/institui.png}}
    \subfigure[Página inicial da Polícia Penal.\label{fig:home_pp}]{\includegraphics[scale=.3]{figure/pp_home.png}}
    \caption{Telas institucionais, O autor}\label{fig:telas_plataforma2}
\end{figure}



Funcionalidades em desenvolvimento:
\begin{itemize}
    \item \textbf{Módulo de Atendimento Municipal} CRAS, CAPS, CREAS, Conselho Tutelar e Secretaria da Saúde
    \item \textbf{Módulo da segurança pública} PM, PP, PC, PCI
    \item \textbf{Integração Avançada entre Apps} Fluxos completos de encaminhamento entre órgãos
    \item \textbf{Notificações e Alertas} Sistema de notificações internas e externas
    \item \textbf{Painel Administrativo Avançado} Customização do admin para fluxos institucionais
    \item \textbf{Testes Automatizados} Cobertura de testes unitários e de integração
    \item \textbf{Aprimoramento de Relatórios} Novos filtros, exportação de dados, relatórios customizados por perfil
    \item Implantação em ambiente de produção e automação de deploy na plataforma \textit{Hostinger}
    \item Sistema de mensagem com o \textit{Django Channels}\footnote{Permite comunicação em tempo real, usado para troca de notificações entre usuários e instituições.}
\end{itemize}


Funcionalidades Pendentes:
\begin{itemize}
    \item Integração com sistemas externos (e-eproc, MP, Defensoria, SISP)
    \item Documentação Técnica Detalhada (Swagger/OpenAPI)
    \item Aprimoramento de UX/UI para acessibilidade
    
    \item Aprimoramento de Segurança, Auditoria de logs, permissões granulares, autenticação de dois fatores
\end{itemize}

\subsection{Assistente virtual com IA}

\par A assistente virtual com inteligência artificial será uma das funcionalidade do projeto, permitindo interações mais naturais e eficientes entre os usuários e a plataforma. Utilizando técnicas de processamento de linguagem natural (PLN) e aprendizado de máquina, o assistente será capaz de compreender e responder a perguntas, fornecer orientações e realizar tarefas automatizadas.
\par Implementado e testado alguns modelos de linguagem com Ollama, algumas figuras de desempenho e respostas obtidas. Frisa-se que, alguns modelos detem de um elevado tempo de respostas isso é devido a restrição de \textit{hardware}.


\begin{figure}[H]
    \center
    \subfigure[Perguntado para a IA.\label{fig:selecao_in}]{\includegraphics[scale=.3]{figure/IA_1/pergunta.png}}
    \subfigure[Resposta final da IA.\label{fig:selecao_out}]{\includegraphics[scale=.3]{figure/IA_1/final_resposta.png}}
    \subfigure[Tempo para resposta.\label{fig:time_response}]{\includegraphics[scale=.3]{figure/IA_1/tempo.png}}
    \caption{Modelo llama3:70b, O autor}\label{fig:modelos_ia}
\end{figure}



\begin{algorithm}[H]
     \KwIn{Gostaria de informar que estou sofrendo, e desejo fazer uma denuncia on line. nao sei fazer a denuncia vou passar os dados e voce faz para eu.} \

    \KwResult{ \\ \textbf{Olá, querida!} \\
    Sinto muito que você esteja passando por essa situação difícil. Eu estou aqui para ajudá-la de qualquer forma possível. \\
    Antes de começar, é importante que você saiba que sua segurança é a minha prioridade. Se você se sentir ameaçada ou em perigo imediato, \textbf{por favor, ligue para o 180 (Central de Atendimento à Mulher) ou 190 (Polícia)}. Eles estarão lá para ajudá-la. \\
    Agora, sobre a denúncia on-line. Eu não posso fazer a denúncia por você, mas posso orientá-la passo a passo. Você pode acessar o site da Delegacia Virtual () e seguir as instruções. \\

    Para começar, você precisará fornecer algumas informações básicas sobre si mesma e sobre o caso de violência doméstica que está enfrentando. Isso inclui: \\

    • Seu nome e contato; \\

    • Detalhes sobre a violência sofrida (data, hora, local, etc.); \\

    • Informações sobre o agressor (nome, idade, endereço, etc.). \\

    É importante que você seja honesta e detalhista ao fornecer essas informações. Elas serão usadas para ajudá-la a obter medidas protetivas e justiça.\\


    Se você precisar de ajuda ou tiver alguma dúvida durante o processo, \textbf{por favor, entre em contato comigo pelo WhatsApp (554832872635)}. Estarei aqui para apoiá-la.\\


    Lembre-se de que você não está sozinha e que há pessoas dispostas a \\
    \texttt{ Resposta gerada por IA local (Ollama - llama3:70b)} \\
     Emergência: 190 | Central da Mulher: 180
    }

    \caption{Interação com o Modelo llama3:70b, tempo de resposta: 224.82 [s]}
   {\fontsize{10pt}{\baselineskip}\selectfont
   Fonte: O autor (2025)}

\end{algorithm}

\par A indagação ao modelo de inteligência artificial necessita para que o chatbot não crie ações que não estejam previstas. Indagou-se a possibilidade de que o chatbot realizasse uma denúncia on line, o modelo respondeu que não poderia fazer a denúncia, mas que poderia orientar a vítima a como fazer. Ressalta-se que o modelo não tem acesso a internet, e todas as respostas são geradas localmente. Ressalto que o tempo de resposta foi de \textbf{224.82 segundos}.

\section{Sugestões/Para implementações futuras}
\par Para aprimoramento do projeto focando na usabilidade, automações e integrações, sugiro algumas implementações:

\begin{itemize}
    \item \textbf{Desenvolvimento de um aplicativo para Polícia Militar}: Desenvolver um aplicativo de forma nativa para dispositivos móveis (Android) que permita aos policiais militares consultar/registrar ocorrências diretamente no local do atendimento. O app deve incluir funcionalidades como captura de fotos, geolocalização automática, preenchimento simplificado - com inserção de informações com áudio -  de formulários e integração direta com a plataforma principal via APIs\footnote{É um conjunto de regras e protocolos que permite que diferentes softwares se comuniquem e troquem informações de maneira padronizada.}.
    
        
    \item \textbf{Predições com Machine Learning}: Utilizar algoritmos de aprendizado de máquina para análise preditiva de risco, identificando padrões de violência e sugerindo intervenções preventivas baseadas em dados históricos, para esta abordagem sugiro a análise e uso da biblioteca \href{https://scikit-learn.org/stable/index.html}{Scikit-learn}, utilizando de seus métodos de predições a exemplo das séries temporais como no modelo \textit{ARIMA}. O framework possiblita uma série de artifícios para a construção de modelos preditivos.
    
    \item \textbf{Segurança Avançada}: Implementar autenticação multifator (MFA) para acesso à plataforma, loggin com a plataforma \textbf{.GOV}, além de auditoria detalhada de logs de atividades dos usuários para monitoramento e conformidade.
    
    \item \textbf{Documentação Técnica Detalhada}: Criar documentação abrangente utilizando Swagger/OpenAPI para todas as APIs desenvolvidas, facilitando futuras integrações e manutenções.
    
    \item \textbf{Testes Automatizados}: Desenvolver uma suíte completa de testes unitários e de integração para garantir a estabilidade e confiabilidade da plataforma em cada nova atualização.
    \item \textbf{Redundâncias de permissões} Implementar nas visualizações críticas verificações adicionais de permissões e usuários autentificados para evitar acessos indevidos. Caso o acesso seja indevido(Falta de login ou qualquer outro) registrar logging detalhado do evento e forcar "ERRO" ``` raise ValeuError("Comentario a ser exibido caso o 'DEBUG' esteja ativo") ```.
    Esta abordagem suprime o erro ao usuário evitando informações para possivél ataque.
\end{itemize}

\section{Conclusão}

\par O projeto PIEVDCS encontra-se em estágio intermediário de desenvolvimento, com cerca de 30\% das funcionalidades previstas já implementadas. Entre os avanços, destacam-se a estrutura básica da aplicação, os modelos de dados principais, a interface inicial e os primeiros dashboards de visualização.

\par No momento, o foco está na construção dos fluxos de integração entre os órgãos, aprimoramento de velocidades nas consultas, testes e customizações de acordo com os perfis institucionais.

\par Por se tratar de um sistema que será utilizado por múltiplas instituições, sua conclusão depende da validação progressiva por parte dos usuários finais — como operadores do sistema, representantes da segurança pública, órgãos do Judiciário e Órgãos Municipais. A metodologia adotada é o Scrum, com entregas em sprints iterativos. No entanto, \textbf{não há um prazo final fixo para a entrega}, sem extrapolar o período da prestação de contas-, já que cada etapa é condicionada à disponibilidade para reuniões, testes e aprovação por parte dos \textit{stakeholders}\footnote{No contexto do desenvolvimento de sistemas, \textit{stakeholders} são todas as pessoas, grupos ou organizações que têm interesse direto ou indireto no projeto, incluindo usuários finais, clientes, desenvolvedores, patrocinadores, gestores e demais partes envolvidas no processo de criação, aprovação e uso do sistema.} (Pessoas Interresadas).

\par No tocante à assistente virtual com inteligência artificial, foram realizados testes iniciais com modelos de linguagem de grande escala (LLMs) utilizando a plataforma Ollama. Embora os resultados sejam promissores, a integração completa dessa funcionalidade na plataforma principal ainda está em fase de planejamento e desenvolvimento.

\par Este projeto é um \textbf{projeto piloto}, desenvolvido para testar a plataforma em um ambiente controlado e obter feedback prático das instituições envolvidas. Essa etapa inicial é fundamental para ajustar funcionalidades, identificar melhorias e garantir que a solução atenda efetivamente às demandas antes de uma possível expansão para outras regiões.

\par Além de seu impacto direto no enfrentamento à violência de gênero, o projeto promove inclusão e ressocialização, ao contar com a atuação técnica de um reeducando em ambiente supervisionado. A proposta comprova o potencial transformador da tecnologia aliada à reintegração social.
